\documentclass{article}
\usepackage{amsmath}
\usepackage{enumitem}
\usepackage{amssymb}
\usepackage{fullpage}
\title{Calculus on Manifolds Chapter 1 Problems}
\author{Brennan Vincent}
\begin{document}
\maketitle
\begin{enumerate}[label=\textbf{1-\arabic*.}]
\item By basic algebra,
\begin{align*}
	\left(\sum_{i=1}^n|x_i|\right)^2 &= \sum_{i=1}^n\sum_{j=1}^n|x_i||x_j|\\
		&= \left(\sum_{i=1}^n x_i^2\right) + 2\sum_{i=1}^{n-1}\sum_{j=i+1}^n |x_i||x_j|\\
		&\geq \sum_{i=1}^n x_i^2\\
		&= |x|^2.
\end{align*}
The result follows by taking square roots.
\item
	From the proof of Theorem 1-1(3), it is clear that equality holds if and only if $\sum_{i=1}^n x^i y^i = |x|\dot|y|$. By Theorem 1-1(2), this is true if and only if $x$ and $y$ are linearly dependent and $\sum_{i=1}^n x^i y^i$ is nonnegative, which happens if $x=\lambda y$ with $\lambda \geq 0$. Geometrically: if $x$ and $y$ point in the same direction.
\item By the problem above, $|x-y|=|x+(-y)|=\leq |x| + |-y| = |x| + |y|$, with equality holding when $x=-\lambda y$ with $\lambda \geq 0$.
\item
\begin{align*}
||x|-|y||^2 &= |x|^2-2|x||y|+|y|^2\\
	&\leq |x|^2 - 2\langle x,y\rangle + |y|^2 &\text{by Theorem 1-1(2)}\\
	&= \langle x,x\rangle - 2\langle x,y\rangle + \langle y,y\rangle\\
	&= \langle x-y,x-y\rangle\\
	&= |x-y|^2.
\end{align*}
\item $|z-x| = |(z-y) + (y - x)| \leq |z-y|+|y-x|$, by Theorem 1-1(3). Geometrically, this means that a straight line between two points is no longer than any other piecewise linear path between them.
\item
\begin{enumerate}
\item If $0=\int_a^b (f-\lambda g)^2$, then $f=\lambda g$ except possibly on a set of measure zero. Then 
\begin{align*}
\left|\int_a^b fg\right| &= \left|\lambda\right|\int_a^b g^2\\
	&= \left|\lambda\right|\left(\int_a^b g^2 \int_a^b g^2\right)^{1/2}\\
	&= \left|\lambda\right|\left(\frac{\int_a^b f^2}{\lambda^2} \int_a^b g^2\right)^{1/2}\\
	&= \left(\int_a^b f^2\right)^{1/2}\left(\int_a^b g^2\right)^{1/2}.
\end{align*}
On the other hand, if $0 < \int_a^b (f-\lambda g)^2$ for all $\lambda$, then $a\lambda^2 + b\lambda + c > 0$ for all $\lambda$, where $a=\int_a^b g^2$, $b=-2\int_a^b fg$, and $c=\int_a^b f^2$. Since the aforementioned quadratic equation in $\lambda$ is never zero, its determinant $b^2-4ac$ must always be negative. Then $b^2 < 4ac$; that is, $4\left(\int_a^b fg\right)^2 < 4 \left(\int_a^b f^2\right)\left(\int_a^b g^2\right)$, and the result follows by dividing by $4$ and taking square roots. (The fact that this inequality is strict shall be important in the next subproblem).
\item Clearly equality does not imply $f=\lambda g$, since we may change one point of $f$ without changing any of the integrals involved.
If however, $f$ and $g$ are continuous, equality \textit{does} imply $f=\lambda g$ for some $\lambda\in \mathbb R$. For if $f$ and $g$ are continuous and $f\neq \lambda g$, then $f\neq \lambda g$ on some open interval, and so $\int_a^b (f-\lambda g)^2 > 0$. But we have already shown in the previous subproblem that this condition holding for all $\lambda$ implies $\left|\int_a^b fg\right| < \left(\int_a^b f^2\right)^{1/2}\left(\int_a^b g^2\right)^{1/2}$.
\item Let $a=0$ and $b=n$. If $i-1\leq x < n$, let $f(x) = x_i$ and $g(x) = y_i$. The result follows.
\end{enumerate}
\item
\begin{enumerate}
\item Assume $T$ is norm preserving. Then
\begin{align*}
\langle Tx,Ty\rangle &= \frac{|Tx+Ty|^2-|Tx-Ty|^2}{4}\\
	&= \frac{|T(x+y)|^2-|T(x-y)|^2}{4}\\
	&= \frac{|x+y|^2 - |x-y|^2}{4}\\
	&= \langle x, y \rangle,
\end{align*}
so $T$ is inner product preserving. Conversely, if $T$ is inner product preserving,
\begin{align*}
|Tx| &= \sqrt{\langle Tx,Tx\rangle}\\
	&= \sqrt{\langle x,x\rangle}\\
	&= |x|.
\end{align*}
\item Assume $T$ is norm preserving. If $Tx = Ty$, then $T(x-y) = 0$, so $|x-y|=|T(x-y)|=0$. Thus $x=y$ by Theorem 1-1(1), and so $T$ is injective.

If $u\in T(\mathbb R^n)$, then $u = Tx$ for some $x$. Then $|T^{-1}u| = |T^{-1}(Tx)| = |x| = |Tx| = |u|$, since $T$ is norm-preserving. Thus $T^{-1}$ is norm-preserving.
\end{enumerate}
\item
\begin{enumerate}
\item Assume that $T$ is norm-preserving. Then $T$ is injective by 1-7(b). Also, $\angle(Tx,Ty) = \arccos\frac{\langle Tx,Ty\rangle}{|Tx||Ty|} = \arccos\frac{\langle x,y\rangle}{|x||y|} = \angle(x,y)$, where we use 1-7(a) to justify the equality $\langle Tx,Ty\rangle = \langle x,y\rangle$.
\item The statement is false. For example, take $x_1 = (1,0)$, $x_2 = (1,1)$, $\lambda_1 = 1$, and $\lambda_2 = -1$. The linear transformation so determined does not preserve angles. (Does the author intend the word \textit{basis} to imply some additional conditions?)
\end{enumerate}
\item We show that $T$ is norm-preserving. Write $u=(x, y)$. Then $|Tu| = \sqrt{(x\cos\theta+y\sin\theta)^2+(-x\sin\theta + y\cos\theta)^2} = \sqrt{x^2+y^2} = |u|$, where we have used the fact that $\cos^2\theta + \sin^2\theta = 1$. Since $T$ is norm preserving, it is angle preserving by 1-8(a).

The second part is a simple computation:
\begin{align*}
\angle(u,Tu) &= \arccos\frac{\langle u,Tu\rangle}{|u||Tu|}\\
	&= \arccos\frac{x^2\cos\theta + xy\sin\theta - xy\sin\theta + y^2\cos\theta}{|u|^2}\\
	&= \arccos\frac{|u|\cos\theta}{|u|}\\
	&= \arccos\cos\theta.
\end{align*}
Since $0 \leq \theta < 2\pi$, $\arccos\cos\theta = \theta$.
\item Let $M'$ be such that for all $i,j$, $T_{i,j} \leq M'$. Let $R_i$ denote the $i$th row of $T$. Then
\begin{align*}
|Th|&=\left(\sum_{i=1}^n \langle R_i,h\rangle^2\right)^{1/2}\\
	&\leq \left(\sum_{i=1}^n |R_i|^2|h|^2\right)^{1/2} &\text{by Theorem 1-1(2)}\\
	&\leq \left(\sum_{i=1}^n mM'^2|h|^2\right)^{1/2}\\
	&= \sqrt{mn}M'|h|.
\end{align*}
The result follows by letting $M=\sqrt{mn}M'$.
\item Trivial.
\item For any $x$, $x'$, and $y$ in $\mathbb R^n$, and $a$ in $\mathbb R$, $T(ax+x')(y) = \langle ax+x',y\rangle = a\langle x,y\rangle + \langle x',y\rangle = (aTx + Tx')(y)$. Thus $T$ is a linear transformation. If $T(x-x')=0$, then $|x-x'|^2 = \langle x-x',x-x'\rangle = T(x-x')(x-x') = 0$, and so $x=x'$. Thus $T$ is injective.

It is a basic theorem of linear algebra that an injective linear transformation on a finite-dimensional space must be a bijection, which amounts to the conclusion we have been asked to prove.
\item If $x$ and $y$ are orthogonal, then $|x+y|^2 = \langle x+y,x+y\rangle = \langle x,x\rangle + 2\langle x,y\rangle + \langle y,y\rangle = |x|^2 + 2\langle x,y\rangle + |y|^2 = |x|^2 + |y|^2$.
\item Let $\mathcal{O}$ be a family of open sets. If $x\in\bigcup \mathcal O$, then $x\in U$ for some $U\in \mathcal O$. Then there is an open rectangle $A$ with $x\in A\subset U \subset \bigcup\mathcal O$, so $\bigcup\mathcal O$ is open.

If $U_1$ and $U_2$ are open, and $x\in U_1\cap U_2$, then $x\in A_1\cap A_2$, where $A_1$ and $A_2$ are open rectangles contained in $U_1$ and $U_2$, respectively. But the intersection of any two non-disjoint open rectangles is an open rectangle. Thus $U_1\cap U_2$ is open.

Now consider the family $\{n\in\mathbb N: U_n\}$, where $U_n = (-\frac{1}{n},\frac{1}{n})$. Its intersection is $\{0\}$, which is not open.
\item Let $x$ be such that $|x-a|<r$. Let $B=\frac{r-|x-a|}{\sqrt{n}}$. Then $B$ is positive. Write $x=(x_1, \ldots, x_n)$, and consider the open rectangle $U=((x_1-B,x_1+B)\times\cdots\times (x_n-B,x_n+B)$. If $y\in U$, then each $|y_i - x_i| < B$, so 
\begin{align*}
|y-x| &= \sqrt{\sum_{i=1}^n (y_i - x_i) ^2}\\
	& < \sqrt{\sum_{i=1}^n B^2}\\
	& \leq r - |x-a|.
\end{align*}
Thus $|y-x| + |x-a| < r$ and so $|y-a| < r$, and $U\subset \{x\in\mathbb R^n:|x-a|<r\}$. Thus the latter set is open.
\item
\item
\item
\item Let $x\in[0,1]$ be irrational. If $x\notin A$, then since $A$ is closed, we may choose some open interval around $x$ containing no points of $A$. Then the intersection of that interval with $[0,1]$ contains no rational numbers, which is absurd.
\end{enumerate}
\end{document}
