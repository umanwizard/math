\documentclass{article}
\usepackage{amsmath}
\usepackage{enumitem}
\usepackage{amssymb}
\usepackage{fullpage}
\newcommand{\li}[2][n] {
  \liminf_{#1\to\infty} #2 
}
\newcommand{\ls}[2][n] {
  \limsup_{#1\to\infty} #2
}

\title{Principles of Mathematical Analysis Chapter 3 Exercises}
\author{Brennan Vincent}
\begin{document}
\maketitle
\begin{enumerate}[label=\textbf{\arabic*.}]
\setcounter{enumi}{1}
\item $\frac{1}{2}$ (proof omitted, can be found on math.stackexchange).
\item We prove by induction that every $s_n < 2$. Since any monotonic bounding sequence converges, $(s_n)$ converges.

The statement is clearly true for $n=1$. If it is true for some $n > 1$, then \[s_{n+1}^2 = 2+\sqrt{s_n} < 2 + \sqrt{2} < 4,\] and so $s_{n+1} < 2$ by taking square roots, since each $s_{n+1}$ is positive.
\item By induction, each $s_{2m} = \frac{1}{2} - \frac{1}{2^m}$, and each $s_{2m+1} = 1 - \frac{1}{2^m}$. Thus $\lim_{m\to\infty} s_{2m} = \frac{1}{2}$, and $\lim_{m\to\infty} s_{2m+1} = 1$. If any other subsequence of $(s_n)$ contains infinitely many elements of each of these subsequences, then it must diverge. Thus any convergent subsequence of $(s_n)$ is equal to one of the subsequences $(s_{2m})$ or $(s_{2m+1})$ except at finitely many points, and therefore converges either to $\frac{1}{2}$ or to $1$. Thus $\li s_n = \frac{1}{2}$ and $\ls s_n = 1$.
\item
\item In each of the following, let $A_n$ denote the $n$th partial sum of the series.
\begin{enumerate}
\item $\sum a_n$ is a telescoping series. Thus $A_n = \sqrt{n+1} - 1$, which diverges.
\item Adding and subtracting $\frac{\sqrt{k+1}}{k+1}$ to each term and rearranging,
\begin{align*}
A_n &= \left(\sum_{k=1}^n \frac{\sqrt{k+1}}{k} - \frac{\sqrt{k+1}}{k+1}\right) + \left(\sum_{k=1}^n \frac{\sqrt{k+1}}{k+1} - \frac{\sqrt k}{k}\right)\\
&= \left(\sum_{k=1}^n\frac{\sqrt{k+1}}{k(k+1)}\right) + \frac{\sqrt{n+1}}{n+1} - 1&\text{since the second sum above is telescoping}.
\end{align*}
Now $\frac{\sqrt{n+1}}{n+1}$ converges to $0$, and \[\frac{\sqrt{k+1}}{k(k+1)} < \frac{1}{(k+1)^{3/2}},\] thus the series of those terms converges by the comparison test and by Theorem 3.28. Thus $(A_n)$ converges.
\item $\sqrt[n]{|a_n|} = \sqrt[n]{n} - 1$, which converges to $0$. Thus $(A_n)$ converges, by the root test.
\item
If $|z| < 1$, \[|a_n| = \frac{1}{|1+z^n|} \geq \frac{1}{2},\] so $(a_n)$ does not converge, and neither does $(A_n)$.

If $|z| = 1$, the same proof goes through, with the caveat that the series is only defined when $z^n + 1 \neq 0$ for all $n$.

If $|z| > 1$, the series converges. We first show that $\frac{z^n}{1+z^n}$ is bounded. By the triangle inequality, \[|z^n| = |z^n + 1 - 1| \leq |z^n + 1| + 1,\] and so $|z^n + 1| \geq |z^n| - 1$. Thus \[\frac{|z^n|}{|1+z^n|} \leq \frac{|z^n|}{|z^n| - 1} = \frac{1}{1 - \frac{1}{|z^n|}}.\] The latter expression is obviously bounded.

Now \[\left|\frac{1}{1+z^n}\right| = \frac{1}{|z|^n}\left|\frac{z^n}{1+z^n}\right| \leq \frac{1}{|z|^n} M,\] for some fixed $M$. The proof then follows from the fact that $|z| > 1$ and from the comparison test.
\end{enumerate}
\setcounter{enumi}{6}
\item We will show that each $\frac{\sqrt{a_n}}{n} \leq a_n + \frac{1}{n^2}$. Since $\sum a_n$ and $\sum \frac{1}{n^2}$ converge, $\sum \left(a_n + \frac{1}{n^2}\right)$ must converge, and so $\sum\frac{\sqrt{a_n}}{n}$ must converge, by the comparison test.
Now to show the assumed fact:\begin{align*}
\left(\frac{\sqrt{a_n}}{n}\right)^2 &= \frac{a_n}{n^2}\\
	&\leq a_n^2 + 2\frac{a_n}{n^2} + \frac{1}{n^4}\\
	&= \left(a_n+\frac{1}{n^2}\right)^2 ;
\end{align*}
now take the square root of each side.
\item Since $(b_n)$ is monotonic and bounded, there exists some $b$ such that $\lim_{n\to\infty} b_n = b$. If $(b_n)$ is decreasing, let $c_n = b_n - b$; otherwise, if $b_n$ is increasing, let $c_n = -b_n - b$. Then $c_n$ is decreasing and $\lim_{n\to\infty} c_n = 0$. Thus $\sum a_n c_n$ converges, by Theorem 3.42. Then \[\sum a_n b_n = \pm \sum a_n (c_n + b) = \pm \left(b\sum a_n + \sum a_n c_n\right),\] which converges.
\setcounter{enumi}{10}
\item \begin{enumerate}
\item If $a_n\not\to 0$, let $\epsilon > 0$ be such that $a_n > \epsilon$ for infinitely many values of $n$. Then \[\frac{a_n}{1+a_n} = \frac{1}{1/a_n+1} > \frac{1}{\epsilon+1}\] for infinitely many values of $n$. Thus $\frac{a_n}{1+a_n}\not\to 0$, and the series diverges.

On the other hand, assume that $a_n\to 0$. If $\sum\frac{a_n}{1+a_n}$ converges, then so does $\sum\frac{a_n^2}{1+a_n}$, by the comparison test. But then \[\sum a_n = \sum\left(\frac{a_n}{1+a_n} + \frac{a_n^2}{1+a_n}\right) = \sum\frac{a_n}{1+a_n} + \sum\frac{a_n^2}{1+a_n}\] converges, a contradiction.
\item \begin{align*}
1 - \frac{s_N}{s_{N+k}} &= \frac{s_{N+k}-s_N}{s_{N+k}}\\
	&= \frac{\sum_{j=1}^k a_{N+j}}{s_{N+k}}\\
	&= \sum_{j=1}^k \frac{a_{N+j}}{s_{N+k}}\\
	&\leq \sum_{j=1}^k \frac{a_{N+j}}{s_{N+j}},
\end{align*}
where the last step holds because $a_n > 0$ for all $n$, and so the sequence $(s_n)$ of partial sums is increasing.

Holding $N$ fixed and letting $k\to\infty$, $\frac{S_N}{S_{N+k}}\to 0$ since $\sum a_n$ diverges. Thus we may choose $k$ large enough such that $\sum_{j=1}^k \frac{a_{N+j}}{s_{N+j}} > \frac{1}{2}$. Since this holds for \textit{any} $N\in\mathbb N$, the Cauchy condition for the partial sums of $\sum \frac{a_n}{s_n}$ cannot be satisfied, and so the series diverges.
\item Since each $a_n > 0$, each $\frac{s_n}{s_{n-1}} > 1$. Then
\begin{align*}
a_n &< \frac{s_n}{s_{n-1}} a_n\\
	&= \frac{s_n}{s_{n-1}}(s_n - s_{n-1})\\
	&= \frac{s_n^2 - s_n s_{n-1}}{s_{n-1}}\\
	&= \frac{s_n^2}{s_{n-1}} - s_n.
\end{align*}
Dividing both sides by $s_n^2$ yields the desired inequality.

Now \[\sum_{n=2}^k \frac{1}{s_{n-1}} - \frac{1}{s_n}\] is the $k$th partial sum of a telescoping series; its value is \[\frac{1}{a_1} - \frac{1}{s_k}.\] Since $s_k\to\infty$, this series converges. Then $\sum\frac{a_n}{s_n^2}$ converges by the comparison test.
\end{enumerate}
\end{enumerate}
\end{document}
