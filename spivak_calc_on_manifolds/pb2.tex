\documentclass{article}
\usepackage{amsmath}
\usepackage{enumitem}
\usepackage{amssymb}
\usepackage{fullpage}
\title{Calculus on Manifolds Chapter 2 Problems}
\author{Brennan Vincent}
\begin{document}
\maketitle
\begin{enumerate}[label=\textbf{2-\arabic*.}]
\item Let $\lambda$ be the derivative of $f$ at $a$. Given $\epsilon > 0$, choose $\delta>0$ such that $|h|<\delta$ implies \[\frac{|f(a+h)-f(a)-\lambda(h)|}{|h|} < \epsilon.\] By 1-10, choose $M$ such that $|\lambda h| \leq M|h|$ for all $h$. Let $\delta' = \min(\frac{\epsilon}{\epsilon+M},\delta)$. Then $|h| < \delta'$ implies
\begin{align*}
|f(a+h) - f(a)| &\leq |f(a+h) - f(a) - \lambda h| + |\lambda h|\\
	&= |h|\frac{|f(a+h)-f(a)-\lambda h|}{|h|} + |\lambda h|\\
	&< |h|\epsilon + M|h|\\
	&= |h|(\epsilon + M)\\
	&< \epsilon.
\end{align*}
\item If $f$ is independent of the second variable, define $g:\mathbb R\to\mathbb R$ by $g(x) = f(x,0)$. Then $f(x,y) = g(x)$ for any $x,y\in\mathbb R$. Conversely, if there is some such $g$, then for any $x, y_1, y_2\in \mathbb R$, $f(x,y_1) = g(x) = f(x, y_2)$.

Given $a,b\in\mathbb R$, let $\lambda:\mathbb R^2\to\mathbb R$ be the linear transformation whose matrix is $(g'(a), 0)$. Given $\epsilon >0$, let $\delta > 0$ be such that $h\in\mathbb R$, $|h|<\delta$ implies \[\frac{|g(a+h)-g(a)-hg'(a)|}{|h|}<\epsilon.\] Then $|(h,k)| < \delta$ implies \[\frac{|f(a+h,b+k)-f(a,b) - \lambda(h,k)|}{|(h,k)|} = \frac{|g(a+h) - g(a) - hg'(a)|}{|(h,k)|} \leq \frac{|g(a+h) - g(a) - hg'(a)|}{|h|} < \epsilon,\] and so $\lambda$ is the derivative of $f$.
\item Constant functions.
\item 
\end{enumerate}
\end{document}
